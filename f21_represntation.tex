\documentclass{amsart}
\usepackage{amsmath}
\usepackage{amssymb}
\usepackage{amsthm}
\usepackage{tikz-cd}
\def\exp{{\rm exp}}
\def\R{\mathbb{R}}
\def\Z{\mathbb{Z}}
\def\Q{\mathbb{Q}}
\def\C{\mathbb{C}}
\def\N{\mathbb{N}}
\def\T{\mathbb{T}}
\def\Y{\mathbb{Y}}
\def\Sympl{\rm Sympl}
\def\Fix{\rm Fix}
\def\char{\rm char}
\author{M. Tikhonov}
\begin{document}

\newtheorem{definition}{Definition}
\newtheorem{note}{Note}
\newtheorem{lemma}{Lemma}

\section{Motivation}

Representation theory, informally, is a way to transfer properties of a group $G$  into the language of linear algebra. 
It started as study of symmetries -- as discribtion of rotation and refectional matrices in euclidean, spherical and hyperbolic geometry. 
Right now, one of the biggest applications in physics is either the symmetry of the system to some gauge transformations (charge, color, etc) or to geometric setup. 
The symmetry plays a huge role in physics, allowing to simplify the calculations and express equations more generally.

But representation theory appeard to be useful in diferent topics of pure mathematics.  
In 1904, Burnisde proved his famous lemma about simplicity of a group of order $p^q q^b$ using results from finite group representation theory.

Later in algebraic number theory we have representation theory appearing in Galois theory since extensions of $\mathbb{Q}$ are descbrided by finite groups.

Representation theory also has a lot of applications in PDEs and ODEs as the set of solutions of the equations form a representation of the group symmetry group.

Following physics, applications representation theory of geometric symmetries (mostly character theory) is very useful in chemstry to classify and descibe moleculas.

Theory also applies to Lie Groups, which is very interesting object as it is both geometrical object (as manifold) and algebraic (as group).

So I would generalize above to two main reasons -- it helps understands the structure of a group or some class of the groups and it appear in wide a variety of contexts.
As a lot of algebraic structure it's a way to get an abstract way of some property accuring in different problems.
\section{Introduction}

Suppose $G$ is a \textit{finite} group and $V$ to be a vector space over a field $k$. We mainly will focus on the case $k=\mathbb{C}$.

\begin{definition}
    A linear representation of $G$ of G in vector space $V$ is homomorphism $\rho: G \to GL(V), g \mapsto \rho(g)$ or $\rho_g$ for simplicity.
\end{definition}

$V$ then is called representation space of $G$. The representation is the discribed pair $(V, \rho)$.

As most definitions in mathematics besides the object we also descbire morphism structure on new class of objects.
\begin{definition}
    

 Representation homoprhism $T$ between $(V, \rho)$ and $(W, \tau)$ is vector space homomoprhism that make following diagram commute:

\end{definition}

\[\begin{tikzcd}
	V && V \\
	\\
	W && W
	\arrow["{\rho(g)}", from=1-1, to=1-3]
	\arrow["T", from=1-3, to=3-3]
	\arrow["T"', from=1-1, to=3-1]
	\arrow["{\tau(g)}"', from=3-1, to=3-3]
\end{tikzcd}\]
If we also know that $T$ is isomorphism of vector spaces, we call it \textit{representation isomorphism}. 

Assuming that $V$ is $n-$dimensional over $\mathbb{C}$ we can pick a basis $v_1, v_2, v_3,\dots, v_n$ and a complex $n-$dimensional representation of $G$ is homomoprhism $\rho: G \to GL_n (\mathbb{C})$.

Basic example:  trivial homomorphism that send everything to unity, which we will call trivial homomoprhism. 
To give a more constructive example consider $S_3$ acting on a set $e_1, e_2, e_3$.
As we know $S_3$ has six elements, namely $(), (12), (23), (13), (132), (123)$. 
Besides the trivial representation we also have a very common homomoprhism to $\mathbb{Z}_2$ usually intoduced during a first encounter of $S_n$, namely sign representation.
Assuming $\char(k) \neq 2$, the homomoprhism is not trivial. 


We will talk about representation of $S_n$ specifically futher later in the paper.

Notice that there's another way to look at representation of $S_3$. Pick a basis $e_1, e_2, e_3 \in GL_3 (\mathbb C)$ and consider $\rho (e_i) = e_\sigma(i)$, for $\sigma \in S_3$, i.e. permutation acting on the set of basis elements. In case of sign permutation, that means 
if the permutation is even, we have swap of two basis vectors and a full rotation (including identity) in other case.

In general case for a finite set $X$, we have a natural complex vector space given by a linear combination of the elements of $X$.
The corresponding representation would be $\rho_g (x) = g . x$ (for all $x\in X, g \in G$)  extended to the entire space by linearity.

\begin{note}
    The constructiion above can be generalized to infinite set $X$ if we restrict to only finete non-zero coefficents in linear combinations.
\end{note}

The most common case of this construction would be the case $X = G$ which  we would call regular representation.

\begin{definition}

The representation of $G$ is called is irreducable if the space $V$ is not empty and no nontrivial subspace of $V$ is stable under action of $G$.
    
\end{definition}

\begin{lemma}
    Every one-dimensional representation is irreducable. Any reducable representation is a direct sum of irreducable representations.
\end{lemma}


The irreducable representation can't be a direct sum of any two representation besides $0$ and $V$.

We are now ready to statement a very powerfull statement about irreducable representations.

\begin{lemma}[Schur's lemma]
    Let $(V, \rho_v)$ and $(W, \rho_w)$ be irreducable representations of the same group $G$.
    \begin{enumerate}
        \item If $T$ is homomoprhisms of representations, then it's either isomorphism or zero map.
        \item If $G$ is finite and $T : V \to V$ is homomoprhism then $T = \lambda I$ for some colmplex number $\lambda$.
    \end{enumerate}
\end{lemma}
\begin{proof}
    TBD
\end{proof}

\section{Abelian groups}

\section{Characters}

\section{Burnside's lemma?}

\section{Representations of symmetric group. Youngs table. Symmetric polynomials. Schur-Weyl duality. }

\section{Connection to Galois thory?}



\end{document}