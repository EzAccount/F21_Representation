\documentclass{amsart}
\usepackage{amsmath}
\usepackage{amssymb}
\usepackage{amsthm}
\usepackage{tikz-cd}
\def\exp{{\rm exp}}
\def\R{\mathbb{R}}
\def\Z{\mathbb{Z}}
\def\Q{\mathbb{Q}}
\def\C{\mathbb{C}}
\def\N{\mathbb{N}}
\def\T{\mathbb{T}}
\def\Y{\mathbb{Y}}
\def\Sympl{\rm Sympl}
\def\Fix{\rm Fix}
\def\char{\rm char}
\def\aut{\rm Aut}
\author{M. Tikhonov}
\begin{document}

\newtheorem{definition}{Definition}
\newtheorem{note}{Note}
\newtheorem{lemma}{Lemma}

\section{Motivation}

Representation theory, informally, is a way to transfer properties of a group $G$  into the language of linear algebra. 
It started as study of symmetries -- as discribtion of rotation and refectional matrices in euclidean, spherical and hyperbolic geometry. 
Right now, one of the biggest applications in physics is either the symmetry of the system to some gauge transformations (charge, color, etc) or to geometric setup. 
The symmetry plays a huge role in physics, allowing to simplify the calculations and express equations more generally.

But representation theory appeard to be useful in diferent topics of pure mathematics.  
In 1904, Burnisde proved his famous lemma about simplicity of a group of order $p^q q^b$ using results from finite group representation theory.

Later in algebraic number theory we have representation theory appearing in Galois theory since extensions of $\mathbb{Q}$ are descbrided by finite groups.

Representation theory also has a lot of applications in PDEs and ODEs as the set of solutions of the equations form a representation of the group symmetry group.

Following physics, applications representation theory of geometric symmetries (mostly character theory) is very useful in chemstry to classify and descibe moleculas.

Theory also applies to Lie Groups, which is very interesting object as it is both geometrical object (as manifold) and algebraic (as group).

So I would generalize above to two main reasons -- it helps understands the structure of a group or some class of the groups and it appear in wide a variety of contexts.
As a lot of algebraic structure it's a way to get an abstract way of some property accuring in different problems.

\section{Introduction}
Unless specified other, suppose $G$ is a \textit{finite} group and $V$ to be a vector space over a field $k$. We mainly will focus on the case $k=\mathbb{C}$.

\begin{definition}
    Given a vector space $V$ we will denote all automorphisms $\aut(V)$ with $GL(V)$
\end{definition}

\begin{definition}
    A linear representation of $G$ of G in vector space $V$ is homomorphism $\rho: G \to GL(V), g \mapsto \rho(g)$ or $\rho_g$ for simplicity.
\end{definition}

$V$ then is called representation space of $G$. The representation is the discribed pair $(V, \rho)$.

We shall also define morphism startucre between represntations.

\begin{definition}
 Representation homoprhism $T$ between $(V, \rho)$ and $(W, \tau)$ is vector space homomoprhism $T: V \to W$ that make following diagram commute:
\end{definition}

\[\begin{tikzcd}
	V && V \\
	\\
	W && W
	\arrow["{\rho(g)}", from=1-1, to=1-3]
	\arrow["T", from=1-3, to=3-3]
	\arrow["T"', from=1-1, to=3-1]
	\arrow["{\tau(g)}"', from=3-1, to=3-3]
\end{tikzcd}\]
If $T$ is isomorphism of vector spaces, then induced (by the property above) representation homorphism is \textit{representation isomophism}.

Assume that $V$ is $n-$dimensional vector space over $\mathbb{C}$. We can pick a basis $v_1, v_2, v_3,\dots, v_n$. Then since for finite dimensional spaces we have $GL(n, C) \simeq GL(C^n)$. Then there's natural map $\rho: G\to GL_n$, which maps basis vector to corresponding column.

Basic example:  trivial homomorphism that send everything to unity, which we will call corresponding representation trivial. 
To give a more constructive example consider $S_3$ acting on a set $e_1, e_2, e_3$.
As we know $S_3$ has six elements, namely $\{ (), (12), (23), (13), (132), (123) \}$. 
Besides the trivial representation we also have a very common homomoprhism to $\mathbb{Z}_2$ usually intoduced during a first encounter of $S_n$, namely sign representation.
Assuming $\char(k) \neq 2$, the homomoprhism is not trivial. 


We will talk about representations of $S_n$ specifically later in the paper.

Notice that there's another way to look at representation of $S_3$. Pick a basis $e_1, e_2, e_3 \in GL_3 (\mathbb C)$ and consider $\rho (e_i) = e_\sigma(i)$, for $\sigma \in S_3$, i.e. permutation acting on the set of basis elements. In case of sign permutation, that means 
if the permutation is even, we have swap of two basis vectors and a full rotation (including identity) in other case.

In general case for a finite set $X$, we have a natural complex vector space $\mathbb{C}(X)$ given by a linear combination of the elements of $X$:
$\forall v \in \mathbb{C}(X): v = \sum_{i=1}^n c_i x_i$, where $c_i \in \mathbb{C}, x_i \in X$
The representation corresponding to the action $G$ on $X$ is $\rho_g (x) = g . x$ (for all $x\in X, g \in G$), which can be extended to the vector entire space by linearity.

\begin{note}
    The constructiion above can be generalized to infinite set $X$ in the following way.
    Let $x_i \in X$ be elements of the set, 
    then vector space $\mathbb{C}(X)=\{ v, \left. \right| v = \sum_{i=1}^\infty c_i x_i\}$, where $ \{c\}_1^\infty \in \mathbb{C}, x_i \in X$, 
    but the sequence $c$ only have finitely many non-zero elements.
\end{note}

The most interesting and useful case arrises if we let $X = G$ so $G$ acts on itself.

\begin{definition}
The representation arrising from $G$ acting on itself by construction above is regular representation.
\end{definition}

\begin{definition}

The representation of $G$ is called is irreducable if the space $V$ is not empty and no nontrivial subspace of $V$ is stable under action of $G$.
    
\end{definition}

\begin{lemma}
    Every one-dimensional representation is irreducable. Any reducable representation is a direct sum of irreducable representations.
\end{lemma}


We are now ready to statement a very powerfull statement about irreducable representations.

\begin{lemma}[Schur's lemma]
    Let $(V, \rho_v)$ and $(W, \rho_w)$ be irreducable representations of the same group $G$.
    \begin{enumerate}
        \item If $T: (V, \rho_v) \to (W, \rho_w)$ is representation homomoprhism, then $T=0$ or $T$ is representation isomophism.
        \item If $G$ is finite and $T : (V, \rho_v) \to (V, rho_v) $ is endomorphism (homorphism of presentation to itself) then $T = \lambda I$ for some colmplex number $\lambda$.
    \end{enumerate}
\end{lemma}
\begin{proof} 
    1. Suppose there is a nontrivial representation homomoprhism $f : (V, \rho_v) \to (W, \rho_w)$. Our goal is to prove that $V \not\simeq W$. Consider the subspace: kernel $\ker f$. Since the map is representation homomoprhism for each group element there's an element in kernel, s.t. $f(\rho_v(g) x) = 0$, meaning that
    $\ker f$ is stable under action of $G$, but we require representation to be trivial, thus $\ker = 0$ or $\ker = T$.

    2. Since $T$ is a linear operator, we can consider an eigenvalue $\lambda$ of $T$. Here we use $k = \mathbb{C}$ to ensure that eigenvalue exists. Consider a map $g = f - \lambda \cdot id$, which would be a representation. If $x$ is eigenvector of $f$, then $g x = 0$, and thus $\ker g$ is non trivial. Since we know that 
    $\ker g$ is either 0 or fullspace (since it's a irreducable representation), then $\ker g = V$ and $g$ is trivial, i.e. $f = \lambda I$.
\end{proof}

\section{Abelian groups}

\section{Characters}

\section{Burnside's lemma?}

\section{Representations of symmetric group. Youngs table. Symmetric polynomials. Schur-Weyl duality. }

\section{Connection to Galois thory?}



\end{document}