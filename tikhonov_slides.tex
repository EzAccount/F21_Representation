\documentclass{beamer}
\usepackage{tikz-cd}
\usetheme{Berlin}
\usepackage{amsmath}
\usepackage{amssymb}
\usepackage{amsthm}
\usepackage{tikz-cd}
\def\exp{{\rm exp}}
\def\R{\mathbb{R}}
\def\Z{\mathbb{Z}}
\def\Q{\mathbb{Q}}
\def\C{\mathbb{C}}
\def\N{\mathbb{N}}
\def\T{\mathbb{T}}
\def\Y{\mathbb{Y}}
\def\Sympl{\rm Sympl}
\def\Fix{\rm Fix}
\def\char{\rm char}
\def\aut{\rm Aut}
\def\ker{\rm Ker\ }
\def\im{\rm Im}
\def\id{\rm id}
\author{M. Tikhonov}

\title{Introduction to representation theory}
\author{M. Tikhonov}
\institute{University of Virginia}
\date{December 7}

\begin{document}


\frame{\titlepage}

\begin{frame}{Outline}
    \begin{itemize}
        \item Motivation 
        \item Main definitions and theorems
        \item Characters
        \item Examples
    \end{itemize}
\end{frame}

\begin{frame}{What is representation?}

\end{frame}

\begin{frame}{Formal definition}
    Let $V$ be a vector space over some field $k$. Denote invertible homomoprhism of $V$ with $GL(V)$.

    Let $G$ be a finite group.
    \begin{definition}
        A linear representation of $G$ in $V$ is a pair $(V, \rho)$ where $\rho: G \to GL(V)$ is a group homomoprhism
    \end{definition}

    For convenience we will sometime denote the pair $(V, \rho)$ with $\rho_V$.

    \begin{example}
        Let $\rho_V : V \to GL(V): v \mapsto 1$.
        Clearly that is a representation. We will call it trivial.
    \end{example}

\end{frame}

\begin{frame}[fragile]
    \begin{definition}
        Homomorphism of representations $V, \rho$ and $W, \tau$ is a vector space homomoprhism $V \to W$ making following diagram commute:
    \end{definition}
    \[\begin{tikzcd}
        V && V \\
        \\
        W && W
        \arrow["{\rho(g)}", from=1-1, to=1-3]
        \arrow["T", from=1-3, to=3-3]
        \arrow["T"', from=1-1, to=3-1]
        \arrow["{\tau(g)}"', from=3-1, to=3-3]
    \end{tikzcd}\]
    If $T$ is isomorphism of vector spaces it is also isomorphism of representations.
\end{frame}



\begin{frame}{Subrepresentations}
    \begin{definition}
        Let $V$ be a representation. We call $W$ a subrepresentation of representation $V$ if it's a subspace $W\subset V$ as a vector space and invariant under $G$. That is, for all $g \in G$ and for all $w \in W$ $g.w \in W$.
    \end{definition}

    \begin{lemma}
        Let $W, V$ be two representation and consider $f$ being representation homomoprhism. Then $\ker f \subset V$ and $\im \ f \subset W$ are subrepresentations.

    \end{lemma}

\end{frame}

\begin{frame}{Induced representation}
    Let $G$ be a group, X be a G-homogenous space. Let $L(X)$ be a space of functions on $X$. Then we have representation $T$ on $L(X)$ given as:
    $$\left[T(g) f\right] (x) = f (xg) $$
    Now fix a point and pick a stationary subgroup $H$ of that point. Then
    $$ F(g) = f(x_0 g) $$
    will be a subrepresentation corresponding to a subspace invariant under $H$.

    Now let's generalize this observation. Let $U$ be a representation of some subgroup $H$ in finite-dimensional $V$.
    Consider a space of functions $F$, s.t. $F\ni f: G \to V$, s.t.
    $$F(hg) = U(h) \cdot F(g)$$
    where $h\in H, g \in G$.

    \begin{definition}
        $F$ defined as above called induced representation of $G$ by subgroup $H$.
    \end{definition}
\end{frame}

\begin{frame}{Reducability}
    \begin{definition}
        The representation of $G$ is called irreducible if the space $V$ is not empty and no nontrivial subspace is invariant under $G$. Or equivalently irreducible representation has no nontrivial subrepresentations.
    \end{definition}
    \begin{lemma}
        Any reducible representation is a direct sum of irreducible representation.
    \end{lemma}
\end{frame}

\begin{frame}{Schur's lemma}
    \begin{lemma}[Schur's lemma]
        Let $(V, \rho_v)$ and $(W, \rho_w)$ be irreducible representations of the same group $G$.
        \begin{enumerate}
            \item If $T: (V, \rho_v) \to (W, \rho_w)$ is representation homomoprhism, then $T=0$ or $T$ is representation isomophism.
            \item If $G$ is finite and $T : (V, \rho_v) \to (V, \rho_v) $ is endomorphism (homorphism of presentation to itself) then $T = \lambda I$ for some colmplex number $\lambda$.
        \end{enumerate}
    \end{lemma}

\end{frame}


\begin{frame}{"Average" on cube problem}
\begin{example}
    Pick a die. Each step we replace the number on each side with average of neighbor sides.
What would be the numbers written after long time?
\end{example}
\end{frame}







\begin{frame}{Cube transformation group}
    \begin{lemma}
        Group of symmetries that preserve cube is isomorphic to $S_4$.
    \end{lemma}
    Now let's fix one face on the cube. Consider subgroup of $S_4$ that also preserve that side is Z/4Z.

    Now define $V_r$ to be a space of all functions defined on our cube, which is $6-$ dimensional real space. Now pick trivial representation of $Z/4Z$ in $V_r$ and induce a representation $T$ on $G$.
\end{frame}

\begin{frame}{Solution via representations}
    Consider an operator $$[Lf] (x) = \frac{1}{4}\sum_{y\in CF} f(y)$$,
    where $x \in CF$ is particular face in the set $CF$ of cube faces, $f$ is some functions on the set of spaces and $L$ is an operator on the space of functions.
    
    \begin{lemma}
        Action of $L$ and action of $G$ commute.
    \end{lemma}

    Let's pay closer attention to representation $T$. Linking index $c(T,T)$ is $3$, meaning that $T$ is sum of $3$ non-trivial representations.    

    Let's denote them as  $T = T_1 \oplus T_2 \oplus T_3$.

\end{frame}


\begin{frame}
    To each subrepresentation there should be a corresponding invariant subspace.

    We can pick invariant subspaces of $V_r$ in the following way: 
    
    $V_1$: constant functions, 
    
    $V_2$: even functions which sums (over cube) to 0 
    
    $V_3$: odd functions.
    
 \end{frame}

 \begin{frame}
    On constant function space one can see that eigenvalue is $\lambda_1=1$. 
    
    For an operator from $V_2$ we can choose a function that is $1$ on the first pair of cube's faces, $0$ on the other, and $-1$ on the last one.
    
    Thus $\lambda_2 = \frac{-1}{2}$. 

    
    For a representative of function $V_3$: fix a face and assign $1$ to it, $-1$ to opposite face, and $0$ to all other faces. 
    
    Thus $\lambda_3 = 0$.

    Clearly in the limit we fall into $V_1$. 

    Now since projection of initial function $f = (1,2,3,4,5,6)$ on $V_1$ is constant function $\frac{21}{6}$, those are desired numbers.
 \end{frame}



\end{document}